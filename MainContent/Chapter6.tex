\chapter{\leavevmode\newline Dissertation Work Plan }
\chaptermark{Heading on Chapter Pages}
\label{chap: Dissertation Work Plan}

\section{Dissertation Progresses Work}


\section*{Phase 1: Preliminary Analysis and Model Development }

\begin{enumerate}
    \item \textbf{Literature Review}:
    \begin{itemize}
        \item Study existing literature on truck-trailer dynamics, reverse parking algorithms, and model-based reinforcement learning applications in vehicle control.
        \item Identify gaps in current methodologies and potential improvements.
    \end{itemize}
    
    \item \textbf{Data Collection}:
    \begin{itemize}
        \item Gather data on truck-trailer dynamics during reverse parking maneuvers. This can include sensor data, camera feeds, and other relevant telemetry.
        \item Use simulators if real-world data collection is challenging or risky.
    \end{itemize}
    
    \item \textbf{Model Development}:
    \begin{itemize}
        \item Develop a dynamic model of the truck-trailer system. This model should capture the kinematics and dynamics of reverse parking.
        \item Validate the model using collected data or through simulation environments.
    \end{itemize}
\end{enumerate}

\section*{Phase 2: Reinforcement Learning Framework and Simulation }
\begin{enumerate}
    \item \textbf{RL Environment Setup}:
    \begin{itemize}
        \item Define the state space (positions, angles, velocities), action space (steering angles, acceleration), and reward function (based on parking accuracy, safety, and maneuver time).
        \item Set up terminal conditions, such as successful parking or collision scenarios.
    \end{itemize}
    
    \item \textbf{Algorithm Selection and Development}:
    \begin{itemize}
        \item Choose an appropriate model-based RL algorithm. Consider algorithms that can efficiently utilize the truck-trailer model to simulate future trajectories.
        \item Implement the algorithm and integrate it with the dynamic model.
    \end{itemize}
    
    \item \textbf{Simulation and Testing}:
    \begin{itemize}
        \item Use a high-fidelity simulator to test the RL agent's performance. Ensure the simulator can accurately replicate real-world parking scenarios.
        \item Iteratively train the agent, refining the model and RL algorithm based on simulation results.
    \end{itemize}
\end{enumerate}

\section*{Phase 3: Comparative Analysis with Other Control Algorithms}
\begin{enumerate}
    \item \textbf{Selection of Control Algorithms}:
    \begin{itemize}
        \item Identify other prominent control algorithms used for truck-trailer reverse parking, such as LQR, MPC, or other traditional methods.
    \end{itemize}
    
    \item \textbf{Implementation and Testing}:
    \begin{itemize}
        \item Implement the selected control algorithms in the simulation environment.
        \item Test each algorithm under similar conditions to ensure a fair comparison.
    \end{itemize}
    
    \item \textbf{Performance Evaluation}:
    \begin{itemize}
        \item Evaluate the performance of each algorithm based on criteria like parking accuracy, safety, time taken, and control smoothness.
        \item Compare the results of the model-based RL approach with the results of the other algorithms.
    \end{itemize}
    
    \item \textbf{Analysis and Reporting}:
    \begin{itemize}
        \item Analyze the strengths and weaknesses of the model-based RL approach in comparison to the other algorithms.
        \item Document findings, insights, and potential areas for improvement or further research.
    \end{itemize}
\end{enumerate}


\section{Deliverables}
This research aims to explore the potential of model-based reinforcement learning (MBRL) in the domain of truck-trailer reverse parking control. The objective is to develop an efficient and robust control strategy that can outperform traditional methods. The deliverables outlined below provide a comprehensive overview of the expected outcomes, tools, and documentation that will be produced during the research process.

\begin{enumerate}
    \item \textbf{Dynamic Model of the Truck-Trailer System}:
    \begin{itemize}
        \item A mathematical representation capturing the kinematics and dynamics of the truck-trailer during reverse parking maneuvers.
        \item Validation results comparing the model's predictions with simulation or real-world data.
    \end{itemize}
    
    \item \textbf{Benchmark Control Algorithms based on Trarditional Algorithms}:
    \begin{itemize}
        \item Dubins path planning for reverse parking path generation
        \item A* path planning for spot search and trajectory generation
        \item LQR controller
        \item MPC controller
    \end{itemize}
    
    \item \textbf{Reinforcement Learning Framework}:
    \begin{itemize}
        \item Detailed documentation of the chosen DQN RL algorithm, including its advantages and limitations.
        \item Detailed documentation of the chosen PPO RL algorithm, including its advantages and limitations.
        \item Detailed documentation of the chosen model-based RL algorithm, including its advantages and limitations.
        \item Source code of the implemented RL algorithm, accompanied by comments and usage instructions.
    \end{itemize}
    
    \item \textbf{Simulation Results}:
    \begin{itemize}
        \item A comprehensive report detailing the performance of the RL agent in various simulated scenarios.
        \item Visual aids such as graphs, plots, and possibly videos showcasing the agent's performance.
    \end{itemize}
    
    \item \textbf{Comparative Analysis}:
    \begin{itemize}
        \item A structured comparison between the MBRL approach and other control algorithms.
        \item Performance metrics, challenges faced, and potential areas of improvement for each method.
    \end{itemize}
    
    \item \textbf{Software Tools and Utilities}:
    \begin{itemize}
        \item Any custom software or utilities developed during the research, including simulation tools, data processing scripts, and visualization aids.
        \item Documentation and user manuals for these tools, ensuring future researchers or collaborators can utilize them effectively.
    \end{itemize}
    
    \item \textbf{Final Thesis}:
    \begin{itemize}
        \item A comprehensive document detailing the entire research process, methodologies, results, and conclusions.
        \item Recommendations for future research or potential real-world applications of the developed MBRL approach.
    \end{itemize}
\end{enumerate}

\section{Research Schedule}

The following is the coursework layout and dissertation schedule separated by milestones and months:

\begin{enumerate}
\item \textbf{Model Development and Validation (September 2023 - December 2023)}
\begin{itemize}
    \item Develop and validate the dynamic model of the truck-trailer system.
    \item Begin preliminary work on the benchmark control algorithms based on traditional methods.
\end{itemize}

\item \textbf{Reinforcement Learning Framework Development (November 2023 - March 2024)}
\begin{itemize}
    \item Implement and test the chosen DQN, TD3, and model-based RL algorithms.
    \item Start simulations and gather initial results.
\end{itemize}

\item \textbf{Comparative Analysis and Refinement (April 2024 - May 2024)}
\begin{itemize}
    \item Conduct a structured comparison between the MBRL approach and other control algorithms.
    \item Refine and optimize the RL algorithms based on the comparative analysis.
\end{itemize}

\item \textbf{Finalization and Thesis Writing (July 2024 - August 2024)}
\begin{itemize}
    \item Finalize all research components, including software tools and utilities.
    \item Write, review, and finalize the PhD thesis.
\end{itemize}
\end{enumerate}
