% \chapter*{Abstract}
\chapter*{\Large \uppercase{Abstract}}
\chaptermark{Abstract}
\label{chap:Abstract}
\addcontentsline{toc}{chapter}{Abstract}

Advanced Driver Assistance Systems (ADAS) are gaining more attention in the automotive industry during the past decades, driven by technological advancements, consumer demand for safety, and regulatory support, and the ultimate goal of autonomous vehicles is to improve the traffic system to achieve higher efficiency, fewer accidents, better energy consumption. In recent years, advancements in sensor technology, hardware, and innovative algorithms have brought the prospect of self-driving vehicles closer to reality, and the market statistics also show consistent growth in ADAS-equipped vehicles, reflecting the industry's commitment to innovation and safety. Simultaneously, there is an escalating need in the transportation industry to enhance efficiency and minimize the environmental footprint associated with the movement of goods and individuals. As a result, numerous prominent companies in the automotive and technology sectors are now focusing their efforts on the creation and refinement of advanced driver assistance systems and autonomous vehicles. 

To mitigate both legal and functional challenges, it is anticipated that high driving automation will first find its application in environments that are both controlled and predictable such as mining fields, harbors and ports, distribution centers and warehouses, agricultural landscapes, and the domain of urban last-mile delivery. Companies like Caterpillar Inc. and John Deere have already marketed autonomous vehicles for mining, farming, and other applications, with collaborations including Carnegie Mellon’s National Robotics Engineering Center \parencite{Koenig_heavy_truck_AD}.

In both commercial and industrial sectors, trailer vehicles have seen substantial growth in popularity, becoming vital in transportation, recreation, and commerce. This trend results from various socioeconomic factors, technological advancements, and regulatory shifts. Statistics from the Trailer Output Report show the global light car trailer market size was valued at USD 1.55 billion in 2021 and is expected to expand at a compound annual growth rate (CAGR) of $3.7\%$ from 2022 to 2030. Smith et al. (2018) also emphasizes the rise of trailers for recreational purposes. However,  the varying weights of trailers significantly alter single-vehicle dynamics, creating instability modes like jackknifing and snaking. Therefore, this research aims to achieve efficient planning and control techniques for truck-trailer wheel robot systems using deep neural networks.

Over the past decades, extensive research efforts have been dedicated to overcoming the intricate challenges associated with TTWR (Truck-Trailer Wheeled Robot) systems. This has resulted in the development of various control algorithms, including PID (Proportional-Integral-Derivative), MPC (Model Predictive Control), fuzzy logic, and other nonlinear control methods, each tailored to address specific complexities within TTWR dynamics. Path planning techniques such as Dubins paths, Reeds-Shepp paths, the A* algorithm, Bezier Curves, and Rapidly-Exploring Random Trees have also been applied to TTWR systems. These have been integrated with different control algorithms to ensure closed-loop stability, with the effectiveness of the proposed solutions being validated through simulations. However, the classical path planner and controller is limited with the tuning Complexity and Lack of Adaptation, which limits the user scenario of TTWR assistance

This research contributes an end-to-end deep reinforcement learning network solution for TTWR reverse autonomous control in low-speed, closed, and unstructured environments, alongside a comparison of classic control methods. The research reviews and contrasts popular motion planning and feedback control frameworks, guaranteeing closed-loop stability and validating proposed solutions via simulations. Path planning comparisons were made among A-star search, Dubins path, and polynomial path, including strategies like linear quadratic (LQ) control and also advanced model predictive control (MPC) techniques to account for physical and sensing limitations. 

The primary goal of the end-to-end DRL controller was to achieve TTWR autonomous reverse driving with a light model based deep reinforcement learning network suitable for deployment on embedded automotive platforms. Although several end-to-end deep neural networks exist for autonomous driving, where the input to the machine learning algorithm are camera images and the output is the steering angle prediction, but traditional neural networks are less robust compared with deep reinforcement learning based control algorithm. Furthermore, the Model-Based DRL potentially improve sample efficiency by allowing an agent to synthesize large amounts of imagined experience. To demonstrate the efficiency, several DRL algorithms are developed with different computational complexity and performance, and these algorithms replace different classical control algorithm units including path planning, control, and then the end-to-end DRL controller is proposed which takes raw sensor data as input, and then generate steering and throttle signal to control the TTWR movement and achieve target chasing, obstacle avoidance and ensurance of passenger comfort at the same time.
